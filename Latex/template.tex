%%%%%%%%%%%%%%%%%%%%%%%%%%%%%%%%%%%%%%%%%
% Twenty Seconds Resume/CV
% LaTeX Template
% Version 1.1 (8/1/17)
%
% This template has been downloaded from:
% http://www.LaTeXTemplates.com
%
% Original author:
% Carmine Spagnuolo (cspagnuolo@unisa.it) with major modifications by
% Vel (vel@LaTeXTemplates.com)
%
% License:
% The MIT License (see included LICENSE file)
%
%%%%%%%%%%%%%%%%%%%%%%%%%%%%%%%%%%%%%%%%%

%----------------------------------------------------------------------------------------
%	PACKAGES AND OTHER DOCUMENT CONFIGURATIONS
%----------------------------------------------------------------------------------------

\documentclass[letterpaper]{twentysecondcv} % a4paper for A4

%----------------------------------------------------------------------------------------
%	 PERSONAL INFORMATION
%----------------------------------------------------------------------------------------

% If you don't need one or more of the below, just remove the content leaving the command, e.g. \cvnumberphone{}

\profilepic{photo.jpg} % Profile picture

\cvname{Aryan Gupta} % Your name
\cvjobtitle{Student} % Job title/career

\cvdate{04 November 1998} % Date of birth
\cvaddress{720 Copper Tree Lane \newline Waxhaw, NC 28173} % Short address/location, use \newline if more than 1 line is required
\cvnumberphone{+1 704-249-1595} % Phone number
\cvsite{http://theguptaempire.net} % Personal website
\cvmail{me@theguptaempire.net} % Email address

%----------------------------------------------------------------------------------------

\begin{document}

%----------------------------------------------------------------------------------------
%	 ABOUT ME
%----------------------------------------------------------------------------------------

\aboutme{Aryan is a student at the University at North Carolina at Charlotte, working towards his BS in Computer Engineering. He loves
Tinkering with hardware, both at the consumer/enterprise level and at the circuit level. He is a self-taught programmer that enjoys
writing code and software and stays up-to-date with C++ advancements. He has had a passion for computers since a very early age. He manages
many servers, both personally and at his place of employment. He loves the outdoors and is active as an Assistant Scoutmaster in his Boy Scout
troop. He loves to be challeneged.} % To have no About Me section, just remove all the text and leave \aboutme{}

%----------------------------------------------------------------------------------------
%	 SKILLS
%----------------------------------------------------------------------------------------

% Skill bar section, each skill must have a value between 0 an 6 (float)
\skills{{C++/5.5},{git/3},{Python/2},{Java/2},{Linux/4}}

%------------------------------------------------

% Skill text section, each skill must have a value between 0 an 6
\skillstext{{Windows/5},{Powershell/3}}

%----------------------------------------------------------------------------------------

\makeprofile % Print the sidebar

%----------------------------------------------------------------------------------------
%	 INTERESTS
%----------------------------------------------------------------------------------------

\section{Objective}
Seeking an internship which provides the opportunity to expand my knowledge as a Computer
Engineer and to gain real world experience. My experience with both hardware and software
ensures that I will make positive contributions to your company.

\section{Education}
\begin{twenty} % Environment for a list with descriptions
	\twentyitem{since 2016}{B.Sc. pending 110/127 credits GPA: 3.4/4.0}{UNC at Charlotte}{Computer Engineering Major \newline Software Systems and Mathmatics Minor}
	\twentyitem{2013-2016}{High school}{Ardrey Kell HS}{GPA: Weighted 4.625     Unweighted 3.5313 \newline Class Rank: 96 of 670}
	\twentyitem{2012-2013}{High school}{Providence Senior HS}{}
\end{twenty}

%----------------------------------------------------------------------------------------
%	 PUBLICATIONS
%----------------------------------------------------------------------------------------

\section{Experience}

\begin{twenty}
	\twentyitem{since Apr. 2018}{Lead Technical Assistant}{}{}
\end{twenty}

%----------------------------------------------------------------------------------------
%	 AWARDS
%----------------------------------------------------------------------------------------

\section{Awards}

\begin{twentyshort} % Environment for a short list with no descriptions
	\twentyitemshort{1987}{All-Time Best Fantasy Novel.}
	\twentyitemshort{1998}{All-Time Best Fantasy Novel before 1990.}
	%\twentyitemshort{<dates>}{<title/description>}
\end{twentyshort}

%----------------------------------------------------------------------------------------
%	 EXPERIENCE
%----------------------------------------------------------------------------------------

\section{Experience}

\begin{twenty} % Environment for a list with descriptions
	\twentyitem{1900}{Alice in Wonderland-The Circra (1900's) Silent Film.}{Film}{The first Alice on film was over a hundred years ago.}
	\twentyitem{1933}{Alice in Wonderland 1933 version.}{Film}{This film stars Ethel griffies and Charlotte Henry. It was a box office flop when it was released.}
	\twentyitem{1951}{Disney Film.}{Film}{Walt Disney brings Lewis Carroll's fantasy story to life in this well done animated classic. Even though many elements from the book were dropped, such as the duchess with the baby pig and mock turtle, this version is without a doubt the most famous Alice adaption made.}
	%\twentyitem{<dates>}{<title>}{<location>}{<description>}
\end{twenty}

%----------------------------------------------------------------------------------------
%	 OTHER INFORMATION
%----------------------------------------------------------------------------------------

\section{Other information}

\subsection{Review}

Alice approaches Wonderland as an anthropologist, but maintains a strong sense of noblesse oblige that comes with her class status. She has confidence in her social position, education, and the Victorian virtue of good manners. Alice has a feeling of entitlement, particularly when comparing herself to Mabel, whom she declares has a ``poky little house," and no toys. Additionally, she flaunts her limited information base with anyone who will listen and becomes increasingly obsessed with the importance of good manners as she deals with the rude creatures of Wonderland. Alice maintains a superior attitude and behaves with solicitous indulgence toward those she believes are less privileged.

%----------------------------------------------------------------------------------------
%	 SECOND PAGE EXAMPLE
%----------------------------------------------------------------------------------------

%\newpage % Start a new page

%\makeprofile % Print the sidebar

%\section{Other information}

%\subsection{Review}

%Alice approaches Wonderland as an anthropologist, but maintains a strong sense of noblesse oblige that comes with her class status. She has confidence in her social position, education, and the Victorian virtue of good manners. Alice has a feeling of entitlement, particularly when comparing herself to Mabel, whom she declares has a ``poky little house," and no toys. Additionally, she flaunts her limited information base with anyone who will listen and becomes increasingly obsessed with the importance of good manners as she deals with the rude creatures of Wonderland. Alice maintains a superior attitude and behaves with solicitous indulgence toward those she believes are less privileged.

%\section{Other information}

%\subsection{Review}

%Alice approaches Wonderland as an anthropologist, but maintains a strong sense of noblesse oblige that comes with her class status. She has confidence in her social position, education, and the Victorian virtue of good manners. Alice has a feeling of entitlement, particularly when comparing herself to Mabel, whom she declares has a ``poky little house," and no toys. Additionally, she flaunts her limited information base with anyone who will listen and becomes increasingly obsessed with the importance of good manners as she deals with the rude creatures of Wonderland. Alice maintains a superior attitude and behaves with solicitous indulgence toward those she believes are less privileged.

%----------------------------------------------------------------------------------------

\end{document}
